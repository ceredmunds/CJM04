\documentclass[12pt]{article}
\usepackage[margin=1in]{geometry}
\geometry{letterpaper}                   
\usepackage{graphicx}
\usepackage{amssymb}
\usepackage{epstopdf}
\DeclareGraphicsRule{.tif}{png}{.png}{`convert #1 `dirname #1`/`basename #1
.tif`.png}

\raggedright % So not straight on both sides
\setlength{\parindent}{0.5in} % So indent paragraphs
\usepackage{indentfirst} % So indent first paragraph after a heading
\usepackage{titlesec}
\usepackage{authblk}
\titleformat{\section}[block]{\normalsize \bfseries \filcenter}{}{1em}{}
\titleformat{\subsection}[block]{\normalsize \bfseries}{}{1em}{}
\titleformat{\subsubsection}[runin]{\bfseries}{}{}{}[]
\titlespacing{\subsubsection}{\parindent}{*2}{\wordsep}
\setcounter{secnumdepth}{0}
\usepackage{setspace}
\usepackage{apacite}
\usepackage{amsmath}
\usepackage{color} % So can change text colour, helpful for editing
\usepackage[usenames,dvipsnames,svgnames,table]{xcolor}
\usepackage{graphicx} % So can add pictures
\usepackage[labelfont=it, labelsep=period]{caption} 
\usepackage{enumitem} % To format lists ok
\usepackage[autostyle]{csquotes}  

\usepackage{fancyhdr} % To add headers
\setlength{\headheight}{15.2pt}
\pagestyle{fancy}
\usepackage{lastpage}
<<<<<<< HEAD
\lhead[CJM04]{CJM04}


\title{PIT experiment with continuous and Pavlovian training}
=======
\lhead[CJM01]{CJM01}


\title{PIT experiment}
>>>>>>> 18ea93b4ee598f15b80c1bc6d07015ae65d806f6
\author{C. E. R. Edmunds}
\date{}% Activate to display a given date or no date

\begin{document}

%\maketitle

\newpage
% \begin{abstract}
% \end{abstract}
\doublespacing

\section{Method}
\subsection{Participants}
<<<<<<< HEAD
The participants were 30 undergraduate psychology students from the University of Plymouth. They were awarded partial course credit for their participation. 

\subsection{Materials}
The experiment was run using PsychoPy \cite{Peirce2007, Peirce2009} on a 
MacBook Pro with a 15-inch screen and responses were recorded with
its keyboard.

There were four food props: a 20g bag of Proper Corn Lightly Sea Salted popcorn, a 25g tub of Tesco home brand cashew nuts, a 40g bag of Cool Original Doritos and a 30g bag of Lightly Salted Kettle chips. These food items were also used in the outcome
=======
The participants were $30$ undergraduate psychology students from the
University of Plymouth. They were awarded partial course credit for their
participation. 

\subsection{Materials}
The experiment was run using PsychoPy \cite{Peirce2007, Peirce2009} on a PC
using a 19 inch display, with a 16:9 aspect ratio. Responses were recorded with
a standard keyboard.

[insert] were used as props. These food items were also used in the outcome
>>>>>>> 18ea93b4ee598f15b80c1bc6d07015ae65d806f6
devaluation procedure. Prior to the experiment, the foods were placed in
separate containers, with the food name on the lid. To devalue foods, cooking
oil (11g) and ground cloves (5g) were mixed into a paste and applied liberally
to the foods. The valued outcome remained unadulterated. 

<<<<<<< HEAD
During the transfer test phase (detailed more fully below), pictures of the
outcomes were used as stimuli. 
=======
%During the transfer test phase (detailed more fully below), pictures of the
%outcomes were used as stimuli. 
Pavlovian stimuli

>>>>>>> 18ea93b4ee598f15b80c1bc6d07015ae65d806f6

\subsection{Procedure}
Before the experiment began, participants were warned that they would be
required to eat several foods during the experiment and that these foods might
taste unpleasant. Then, the food props were presented and the participant was
told that they could earn points for food in the following experiment.

\paragraph{Liking ratings} Participants initially rated their desire to eat
each food in a random order (1=``Not at all'', 7=``Very much''). Following the
ratings, the props were removed. 

<<<<<<< HEAD
\paragraph{Instrumental training} Participants were told that ``You can now
earn the four foods shown before by pressing the left (``Q'') or right (``P'')
keys.'' and that ``Your task is to learn which keys earn each food.'' There were
48 trials. They were also warned that they would have to press each button several times in order to receive a reward. 

For this section, the outcomes were linked in pairs. In other words, at any point the participant could receive $O_1$ or $O_2$ by pressing the right or left keys respectively, or they could receive $O_3$ or $O_4$ by pressing the right or left keys respectively. Of course, the actual outcome given depended on which keys the participant pressed. The instrumental responses were reinforced on a ratio10 schedule, i.e. every single button press had a $1/10$ chance of resulting in the outcome being displayed.

For the participant, ``$\leftarrow$ or $\rightarrow$'' was displayed in
white in the centre of the screen. The participant was then free to press either the left or right key. On each button press, the participant hears a sound (basically, click boing of duration ~1 sec). If that particular response does not result in an outcome, nothing happens and the participant is free to press the button again. If that response does result in an outcome, then feedback was displayed (e.g., ``You win one CRISPS point.'') in white for 3000ms. The inter-trial interval was 500ms.

\paragraph{Instrumental knowledge test}
Immediately following instrumental training, we asked participants if they had explicit knowledge of the training contingencies. They were asked for each outcome, in a random order, ``Which key earned [OUTCOME] points, the left or right key?'' After each forced
choice question, they also rated their confidence on a Likert scale (1=``Not at
all confident'', 7=``Very confident''). 

\paragraph{Pavlovian training}
Here, participants were tasked with learning which of the cues ``H'', ``K'', ``S'' and ``Z'' resulted in points for which food outcome. On each trial, a letter cue was presented for 500ms, then the four food options were added in a vertical list, in a random order, below the letter cue. Participants used the mouse to select which food was predicted by the cue. Each cue-food pair was trained 8 times, resulting in 48 trials. p

=======
\paragraph{Pavlovian conditioning}

\paragraph{Instrumental training} Participants were told that ``You can now
earn the four foods shown before by pressing the left (``E'') or right (``I'')
key.'' and that ``Your task is to learn which keys earn each food.'' There were
48 trials. Each trial began with ``$\leftarrow$ or $\rightarrow$'' displayed in
white in the centre of the screen until the participants pressed either the
right or left key. Each key was selectively paired with two of the possible
outcomes. This was counterbalanced across participants. Two outcomes were
available on each trial, and so each key was associated with its two outcomes
50\% of the time. Following the participant's response, feedback was displayed 
(e.g., ``You win one CRISPS point.'') in white for 3000ms. The inter-trial
interval was 500ms.

\paragraph{Instrumental knowledge test}
Following training, we asked participants if they had explicit knowledge of the
training contingencies. They were asked for each outcome, in a random order,
``Which key earned [OUTCOME] points, the left or right key?'' After each forced
choice question, they also rated their confidence on a Likert scale (1=``Not at
all confident'', 7=``Very confident''). 

>>>>>>> 18ea93b4ee598f15b80c1bc6d07015ae65d806f6
\paragraph{Outcome devaluation}
Participants sampled all four of the outcomes, of which two were devalued. They
were told that the devalued outcomes were out of date (a deception), and that
these were the outcomes now available. Liking ratings were then 

\paragraph{Transfer test}
At the start of the transfer test, participants were told: 

<<<<<<< HEAD
=======
% Edit
>>>>>>> 18ea93b4ee598f15b80c1bc6d07015ae65d806f6
\blockquote{In this part of the task, you can earn the four foods by pressing
the left (``E'') or right (``I'') key in the same way as before. You will only
be told how much of each food you have earned at the end of the experiment.
Also, sometimes pictures of the foods will be presented before you choose the
left or right key. NOTE: You will be required to eat all of the food you have
earned at the end of the experiment so choose carefully.}

<<<<<<< HEAD
At the beginning of each trial, a picture stimulus was shown for 3000ms. The
choice instruction then appeared (``$\leftarrow$ or $\rightarrow$'') until the
participant made their choice. No feedback was given. The inter-trial interval
was 500ms. The 64 trials were split into 8 rounds of 8 trials, with each
stimulus appearing twice in each round in a random order. 
=======
Continuous transfer test


>>>>>>> 18ea93b4ee598f15b80c1bc6d07015ae65d806f6

\paragraph{Knowledge tests}
Finally, participants were asked to complete several knowledge tests to check
they had understood the experiment. First, they completed a second set of
liking ratings. Second, they were asked to judge which of the four outcomes 
were out of date. Confidence ratings were taken after each question. Third,
they completed a second instrumental knowledge test. Finally, they were given a
Pavlovian knowledge test, where each stimulus was shown and the participant was
asked to say which outcome the stimulus represents. 

%\begin{table}[b!]
%	\centering
%	\caption{Training and test summaries} \label{table:experimentSummary}
%	\begin{tabular}{p{2cm} p{2cm} p{0.5cm} p{1.5cm} p{3cm} p{3.5cm}}
%		\hline\noalign{\smallskip}
%		\multicolumn{2}{c}{Training} && \multicolumn{3}{c}{Test}\\
%		\noalign{\smallskip}\cline{1-2} \cline{4-6}\noalign{\smallskip}
%		Element & Compound && Ratings & Forced choice & Implicit association\\
%		\noalign{\smallskip}\hline\noalign{\smallskip}
%		A+ {\it (6)} & A+ {\it (12)} && TX {\it (2)} & TN vs. TX {\it(1)}& Left: X,
%+\\
%		B+ {\it (6)} & B+ {\it (12)} && TN {\it (2)} & N vs. X {\it(1)}& Right: N,
%-\\
%		Z- {\it (6)} & Z- {\it (6)}  && A {\it (2)}&&\\
%					 & T+ {\it (12)}	 && B {\it (2)}&& Left: N, +\\
%					 & N- {\it (48)} && X {\it (2)}&& Right: X, -\\
%					 & AX- {\it (24)}&& N  {\it (2)}&&\\
%					 & BX- {\it (24)}&&&&\\
%					 & AZ+ {\it (12)}&&&&\\
%					 & BZ+ {\it (12)}&&&&\\
%		\noalign{\smallskip}\hline
%		\multicolumn{6}{p{14.5cm}}{\footnotesize Note: A-Z refer to cue types, ``+''
%represents the outcome occurred (hair turned blonde/yellow screen), ``-''
%represents no outcome (hair stayed grey/grey screen), and numbers in
%parentheses denote the number of trials.}
%	\end{tabular}
%\end{table}


\subsection{Analysis} 
All analyses were conducted using R \cite{Rcite} and are available at
\texttt{www.gitlab.com/ceredmunds/CJM04}.


<<<<<<< HEAD
%\section{Results}
%
%\subsection{Instrumental knowledge tests and exclusions}
%One participant failed the Pavlovian knowledge test. Nine participants failed
%the instrumental knowledge tests, 1 in the first test and 8 more in the second.
%With regards to participants' confidence in the instrumental ratings (see Figure~\ref{figure:CJM02confidenceGraph}), there was a significant interaction between test time and whether the cue had been devalued or not, $F(1,28)=4.94$, $\eta^2_g=0.01$, $p=.035$. Participants were more confident immediately after instrumental training than after the transfer test, $F(1,28)=63.46$, $\eta^2_g=0.45$, $p<.001$, and were less confident for devalued cues than valued ones, $F(1,28)=5.14$, $\eta^2_g=0.00$, $p=.031$. 
%The 10 participants that failed the knowledge tests were removed from the
%following analyses.
%
%\begin{figure}
%	\centering
%	\includegraphics[width=.66\textwidth]{Images/CJM02confidenceGraph.pdf}
%	\caption{Mean confidence ratings for instrumental knowledge test. Error bars are 95\% confidence ratings.}
%	\label{figure:CJM02confidenceGraph}
%\end{figure}
%
%\subsection{Liking ratings}
%Mean liking ratings are shown in Figure~\ref{figure:CJM02likingGraph}.
%There was a significant interaction between test time and cue value,
%$F(1,18)=46.77$, $\eta^2_g=0.37$, $p<.001$. There was also a significant main
%effect of test time, $F(1,18)=12.68$, $\eta^2_g=0.11$, $p=.002$, and cue value,
%$F(1,18)=23.51$, $\eta^2_g=0.24$, $p<.001$.
%
%\begin{figure}
%	\centering
%	\includegraphics[width=0.66\textwidth]{Images/CJM02likingGraph}
%	\caption{Mean liking ratings. Ratings of 1 and 7 indicate wanting to eat the
%outcome ``not at all'' and ``very much'' respectively.}
%	\label{figure:CJM02likingGraph}
%\end{figure}
%
%\subsection{Transfer phase}
%Mean preference for response 1 ($R_1$) are shown in Figure~\ref{figure:CJM02transferGraph}.
%
%\begin{figure}
%	\centering
%	\includegraphics[width=0.66\textwidth]{Images/CJM02transferGraph}
%	\caption{Transfer test results. Response choice was tested in the presence of
%stimuli representing each outcome. The dependent variable is the percent choice
%of the R1 response. 0.5 indicate no preference. Scores greater that 0.5
%indicate a preference towards R1 and scores lower that 0.5 indicate a
%preference to R2.}
%	\label{figure:CJM02transferGraph}
%\end{figure}
%
%There was a significant interaction between devaluation and associated response, $F(1,18)=13.79$, $\eta^2_g=0.29$, $p=.002$. As you might expect given the design, there was a significant main effect of response consistency, $F(1,18)=25.23$, $\eta^2_g=0.32$, $p=<.001$; participants pressed $R_1$ more for cues $O_1$ and $O_3$ than for the other two cues. 
%
%Furthermore, there was a significant different in response preference between the valued cues $O_1$ and $O_2$, $t(18)=10.73$, $d=2.45$, $p<.001$, but not for the devalued cues, $t(18)=0.15$, $d=0.04$, $p=.880$. 
%
%\begin{figure}
%	\centering
%	\includegraphics[width=.66\textwidth]{Images/CJM02violinGraph.pdf}
%	\caption{Alternative view of the transfer phase data. The width of the bars gives a measure of density}
%	\label{figure:CJM02violinGraph}
%\end{figure}
%
%Figure~\ref{figure:CJM02violinGraph} gives an insight into why there are no significant differences in response preference between the two devalued cues. Participants appeared to be much more uncertain about their response choice here. From asking them to report their response strategy after training, there appear to be broadly two types of respondent (as well as a few who just button mashed/forgot everything, which would hopefully be excluded here):
%
%\begin{enumerate}
%	\item these participants tried to be ``good'' and complete the task pressing the same buttons for all cues as in the instrumental phase. \\
%	\item these participants tried to avoid the devalued outcomes. For the valued outcomes, they pressed the same keys as in the instrumental training phase. For the devalued outcomes, they pressed the opposite keys in an attempt to avoid receiving the outcome.\\
%\end{enumerate}
=======
\section{Results}
>>>>>>> 18ea93b4ee598f15b80c1bc6d07015ae65d806f6


\section{Discussion}

<<<<<<< HEAD
=======
\subsection{Future directions}

>>>>>>> 18ea93b4ee598f15b80c1bc6d07015ae65d806f6

\newpage
\bibliographystyle{apacite}
\bibliography{References}

\end{document}