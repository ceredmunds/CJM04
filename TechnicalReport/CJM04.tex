\documentclass[12pt]{article}
\usepackage[margin=1in]{geometry}
\geometry{letterpaper}                   
\usepackage{graphicx}
\usepackage{amssymb}
\usepackage{epstopdf}
\DeclareGraphicsRule{.tif}{png}{.png}{`convert #1 `dirname #1`/`basename #1
.tif`.png}

\raggedright % So not straight on both sides
\setlength{\parindent}{0.5in} % So indent paragraphs
\usepackage{indentfirst} % So indent first paragraph after a heading
\usepackage{titlesec}
\usepackage{authblk}
\titleformat{\section}[block]{\normalsize \bfseries \filcenter}{}{1em}{}
\titleformat{\subsection}[block]{\normalsize \bfseries}{}{1em}{}
\titleformat{\subsubsection}[runin]{\bfseries}{}{}{}[]
\titlespacing{\subsubsection}{\parindent}{*2}{\wordsep}
\setcounter{secnumdepth}{0}
\usepackage{setspace}
\usepackage{apacite}
\usepackage{amsmath}
\usepackage{color} % So can change text colour, helpful for editing
\usepackage[usenames,dvipsnames,svgnames,table]{xcolor}
\usepackage{graphicx} % So can add pictures
\usepackage[labelfont=it, labelsep=period]{caption} 
\usepackage{enumitem} % To format lists ok
\usepackage[autostyle]{csquotes}  

\usepackage{fancyhdr} % To add headers
\setlength{\headheight}{15.2pt}
\pagestyle{fancy}
\usepackage{lastpage}

\lhead[CJM04]{CJM04}


\title{Continuous PIT experiment with crossed outcomes}

\author{C. E. R. Edmunds}
\date{}

\begin{document}

\maketitle
\doublespacing
\newpage
	\begin{abstract}
		This experiment examines specific-PIT using a crossed outcome design,
inspired by Experiment 2 of \citeA{Seabrooke2017}, where one response is
devalued. It includes Pavlovian training as well as continuous instrumental
training and transfer test phases. 
	\end{abstract}


\section{Method}
\subsection{Participants}

The participants were approximately $30$ undergraduate psychology students from
the University of Plymouth. They were awarded partial course credit for their

\subsection{Materials}
The experiment was run using PsychoPy \cite{Peirce2007, Peirce2009} on a 
MacBook Pro with a 15-inch screen and responses were recorded with
its keyboard.

[INSERT] were used as props. These food items were also used in the outcome
devaluation procedure. Prior to the experiment, the foods were placed in
separate containers, with the food name on the lid. To devalue foods, cooking
oil (11g) and ground cloves (5g) were mixed into a paste and applied liberally
to the foods. The valued outcomes remained unadulterated. 

In the Pavlovian and transfer phases, the stimuli associated with the outcomes
were the letters `H', `K', `S' and `W', approximately 5cm high,
displayed in white on a black background.

\subsection{Procedure}
Before the experiment began, participants were warned that they would be
required to eat several foods during the experiment and that these foods might
taste unpleasant. Then, the food props were presented and the participant was
told that they could earn points for food in the following experiment.

\paragraph{Liking ratings} Participants initially rated their desire to eat
each food in a random order (1=``Not at all'', 7=``Very much''). Following the
ratings, the props were removed. 

\paragraph{Instrumental training} Participants were told that ``In the
following task, your aim is to learn which key, the left (``Q'') or the right
(``P'') key, results in points for which food.'' and that ``Your task is to
learn which keys earn each food''. They were asked to use the first finger of
their dominant hand to respond and warned that they might need to press each
key multiple times for anything to be displayed. 

In this phase, the outcomes were linked in pairs and each participant
received 48 outcomes each. In other words, at any point the participant could
receive $O_1$ or $O_2$ or they could receive $O_3$ or $O_4$, by pressing the
right or left keys respectively. Of course, the actual outcome given on each
trial depended on which key the participant pressed. The instrumental responses
were reinforced on a ratio10 schedule, i.e. every single button press had a
$1/10$ chance of resulting in the outcome being displayed.

For the participant, ``$\leftarrow$ or $\rightarrow$'' was displayed in
white in the centre of the screen. The participant was then free to press
either the left or right key. On each button press, the participant heard a
button press sound (basically, ``click-boing'' of duration ~1 sec). If that
particular response does not result in an outcome, nothing happened and the
participant was free to press either button again. If that response did result
in an outcome, then feedback was displayed (e.g., ``You win one CRISPS
point.'') in white for 3000ms. The inter-trial interval was 500ms.

\paragraph{Instrumental knowledge test}
Immediately following instrumental training, we asked participants if they had
explicit knowledge of the training contingencies. They were asked for each
outcome, in a random order, ``Which key earned OUTCOME points, the left or
right key?'' for which they had to press the ``Q'' or ``P'' key as appropriate.
After each forced choice question, they also rated their confidence on a Likert
scale (1=``Not at all confident'', 7=``Very confident'') using the number keys.

\paragraph{Pavlovian training}
Here, participants were tasked with learning which of the cues ``H'', ``K'',
``S'' and ``Z'' resulted in points for which food outcome. The cue to food
outcome pairing was randomly assigned for each participant. On each trial, a
letter cue was presented for 500ms, then the four food options were added in a
vertical list, in a random order, below the letter cue. Participants used the
mouse to select which food was predicted by the cue. Each cue-food pair was
trained 16 times, resulting in 96 trials.

\paragraph{Outcome devaluation}
Participants sampled all four of the outcomes, of which two were devalued. They
were told that the devalued outcomes were out of date (a deception), and that
these were the outcomes now available. The outcomes that were devalued were
either those associated with the left key or those associated with the right
key, and this was randomised for each participant. Liking ratings for each food
outcome were then taken again, as before. 

\paragraph{Transfer test}
At the start of the transfer test, participants were told: 

\blockquote{In this part of the task, you can earn the four foods by pressing'
the left (``Q'') or right (``P'') key in the same way as before. You will only
be told how much of each food you have earned at the end of the experiment.
Also, the letter stimuli of the foods will be presented.}

The participants were also reminded that they would be required to eat all of
the food they earned at the end of the experiment so they should chose
carefully. 

In the transfer test, each ``trial'' consisted of 30s of a screen displaying
``$\leftarrow$ or $\rightarrow$'' and 30s where an additional letter stimulus
was added. Each letter stimulus was shown twice. The number of left and right
key presses in each 5 second window were recorded. As in the instrumental
training phase, a sound was played for each button press.

\paragraph{Knowledge tests}
Finally, participants were asked to complete several knowledge tests to check
they had understood the experiment. First, they completed a second set of
liking ratings. Second, they were asked to judge which of the four outcomes 
were out of date. Confidence ratings were taken after each question. Third,
they completed a second instrumental knowledge test. Finally, they were given a
Pavlovian knowledge test, where each stimulus was shown and the participant was
asked to say which outcome the stimulus represents. 

\subsection{Analysis} 
All analyses were conducted using R \cite{Rcite} and are available at
\texttt{www.gitlab.com/ceredmunds/CJM04}.

\newpage
\bibliographystyle{apacite}
\bibliography{References}

\end{document}