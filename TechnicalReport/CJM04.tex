\documentclass[12pt]{article}
\usepackage[margin=1in]{geometry}
\geometry{letterpaper}                   
\usepackage{graphicx}
\usepackage{amssymb}
\usepackage{epstopdf}
\DeclareGraphicsRule{.tif}{png}{.png}{`convert #1 `dirname #1`/`basename #1
.tif`.png}

\raggedright % So not straight on both sides
\setlength{\parindent}{0.5in} % So indent paragraphs
\usepackage{indentfirst} % So indent first paragraph after a heading
\usepackage{titlesec}
\usepackage{authblk}
\titleformat{\section}[block]{\normalsize \bfseries \filcenter}{}{1em}{}
\titleformat{\subsection}[block]{\normalsize \bfseries}{}{1em}{}
\titleformat{\subsubsection}[runin]{\bfseries}{}{}{}[]
\titlespacing{\subsubsection}{\parindent}{*2}{\wordsep}
\setcounter{secnumdepth}{0}
\usepackage{setspace}
\usepackage{apacite}
\usepackage{amsmath}
\usepackage{color} % So can change text colour, helpful for editing
\usepackage[usenames,dvipsnames,svgnames,table]{xcolor}
\usepackage{graphicx} % So can add pictures
\usepackage[labelfont=it, labelsep=period]{caption} 
\usepackage{enumitem} % To format lists ok
\usepackage[autostyle]{csquotes}  

\usepackage{fancyhdr} % To add headers
\setlength{\headheight}{15.2pt}
\pagestyle{fancy}
\usepackage{lastpage}
\lhead[CJM04]{CJM04}


\title{Continuous PIT experiment with crossed outcomes}
\author{C. E. R. Edmunds}
\date{}% Activate to display a given date or no date

\begin{document}

\maketitle
\doublespacing
\newpage
	\begin{abstract}
		This experiment examines specific-PIT using a crossed outcome design,
inspired by Experiment 2 of \citeA{Seabrooke2017}, where one response is
devalued. It includes Pavlovian training as well as continuous instrumental
training and transfer test phases. 
	\end{abstract}


\section{Method}
\subsection{Participants}
The participants were approximately $30$ undergraduate psychology students from
the University of Plymouth. They were awarded partial course credit for their
participation. 

\subsection{Materials}
The experiment was run using PsychoPy \cite{Peirce2007, Peirce2009} on a PC
using a 19 inch display, with a 16:9 aspect ratio. Responses were recorded with
a standard keyboard.

[insert] were used as props. These food items were also used in the outcome
devaluation procedure. Prior to the experiment, the foods were placed in
separate containers, with the food name on the lid. To devalue foods, cooking
oil (11g) and ground cloves (5g) were mixed into a paste and applied liberally
to the foods. The valued outcomes remained unadulterated. 

In the Pavlovian and transfer phases, the stimuli associated with the outcomes
were 


\subsection{Procedure}
Before the experiment began, participants were warned that they would be
required to eat several foods during the experiment and that these foods might
taste unpleasant. Then, the food props were presented and the participant was
told that they could earn points for food in the following experiment.

\paragraph{Liking ratings} Participants initially rated their desire to eat
each food in a random order (1=``Not at all'', 7=``Very much''). Following the
ratings, the props were removed. 

\paragraph{Pavlovian conditioning}

\paragraph{Instrumental training} Participants were told that ``You can now
earn the four foods shown before by pressing the left (``E'') or right (``I'')
key.'' and that ``Your task is to learn which keys earn each food.'' There were
48 trials. Each trial began with ``$\leftarrow$ or $\rightarrow$'' displayed in
white in the centre of the screen until the participants pressed either the
right or left key. Each key was selectively paired with two of the possible
outcomes. This was counterbalanced across participants. Two outcomes were
available on each trial, and so each key was associated with its two outcomes
50\% of the time. Following the participant's response, feedback was displayed 
(e.g., ``You win one CRISPS point.'') in white for 3000ms. The inter-trial
interval was 500ms.

\paragraph{Instrumental knowledge test}
Following training, we asked participants if they had explicit knowledge of the
training contingencies. They were asked for each outcome, in a random order,
``Which key earned [OUTCOME] points, the left or right key?'' After each forced
choice question, they also rated their confidence on a Likert scale (1=``Not at
all confident'', 7=``Very confident''). 

\paragraph{Outcome devaluation}
Participants sampled all four of the outcomes, of which two were devalued. They
were told that the devalued outcomes were out of date (a deception), and that
these were the outcomes now available. Liking ratings were then 

\paragraph{Transfer test}
At the start of the transfer test, participants were told: 

% Edit
\blockquote{In this part of the task, you can earn the four foods by pressing
the left (``E'') or right (``I'') key in the same way as before. You will only
be told how much of each food you have earned at the end of the experiment.
Also, sometimes pictures of the foods will be presented before you choose the
left or right key. NOTE: You will be required to eat all of the food you have
earned at the end of the experiment so choose carefully.}

Continuous transfer test



\paragraph{Knowledge tests}
Finally, participants were asked to complete several knowledge tests to check
they had understood the experiment. First, they completed a second set of
liking ratings. Second, they were asked to judge which of the four outcomes 
were out of date. Confidence ratings were taken after each question. Third,
they completed a second instrumental knowledge test. Finally, they were given a
Pavlovian knowledge test, where each stimulus was shown and the participant was
asked to say which outcome the stimulus represents. 

%\begin{table}[b!]
%	\centering
%	\caption{Training and test summaries} \label{table:experimentSummary}
%	\begin{tabular}{p{2cm} p{2cm} p{0.5cm} p{1.5cm} p{3cm} p{3.5cm}}
%		\hline\noalign{\smallskip}
%		\multicolumn{2}{c}{Training} && \multicolumn{3}{c}{Test}\\
%		\noalign{\smallskip}\cline{1-2} \cline{4-6}\noalign{\smallskip}
%		Element & Compound && Ratings & Forced choice & Implicit association\\
%		\noalign{\smallskip}\hline\noalign{\smallskip}
%		A+ {\it (6)} & A+ {\it (12)} && TX {\it (2)} & TN vs. TX {\it(1)}& Left: X,
%+\\
%		B+ {\it (6)} & B+ {\it (12)} && TN {\it (2)} & N vs. X {\it(1)}& Right: N,
%-\\
%		Z- {\it (6)} & Z- {\it (6)}  && A {\it (2)}&&\\
%					 & T+ {\it (12)}	 && B {\it (2)}&& Left: N, +\\
%					 & N- {\it (48)} && X {\it (2)}&& Right: X, -\\
%					 & AX- {\it (24)}&& N  {\it (2)}&&\\
%					 & BX- {\it (24)}&&&&\\
%					 & AZ+ {\it (12)}&&&&\\
%					 & BZ+ {\it (12)}&&&&\\
%		\noalign{\smallskip}\hline
%		\multicolumn{6}{p{14.5cm}}{\footnotesize Note: A-Z refer to cue types, ``+''
%represents the outcome occurred (hair turned blonde/yellow screen), ``-''
%represents no outcome (hair stayed grey/grey screen), and numbers in
%parentheses denote the number of trials.}
%	\end{tabular}
%\end{table}


\subsection{Analysis} 
All analyses were conducted using R \cite{Rcite} and are available at
\texttt{www.gitlab.com/ceredmunds/CJM04}.


\section{Results}


\section{Discussion}

\subsection{Future directions}


\newpage
\bibliographystyle{apacite}
\bibliography{References}

\end{document}